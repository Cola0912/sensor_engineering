\documentclass{article}
\usepackage{amsmath}
\usepackage{graphicx}
\usepackage{float}  % 画像の位置を強制するためのパッケージ
\usepackage[utf8]{inputenc}  % UTF-8エンコーディング対応
\usepackage{zxjatype}  % 日本語の文字コード対応
\usepackage[ipaex]{zxjafont}  % 日本語フォント
\usepackage[margin=2cm]{geometry}  % 左右の余白を2cmに設定

\title{センサ工学演習問題 2 レポート}
\author{相田 舟星\\学籍番号: 21C1002}
\date{\today}

\begin{document}

% 表紙
\begin{titlepage}
    \centering
    \vspace*{\fill}
    {\huge \textbf{センサ工学演習問題 2 レポート}}\\[1.5cm]
    {\Large 相田 舟星}\\
    {\Large 学籍番号: 21C1002}\\[2cm]
    {\large \today}
    \vspace*{\fill}
\end{titlepage}

\newpage

\section*{1. 次の表に示すデータがある。最小2乗法によって当てはまる直線の方程式を求め、結果を図で示せよ。}
\begin{quote}
    \(X = \{-3, -1, 1, 3\}\) 、 \(Y = \{0, 1, 2, 4\}\)
\end{quote}
\begin{figure}[H]
    \centering
    \includegraphics[width=0.7\linewidth]{1.png}
\end{figure}

\subsection*{回答}
最小二乗法により、直線の方程式 \(y = ax + b\) を求めます。まず、平均 \(\bar{X}\) と \(\bar{Y}\) を計算します。

\[
\bar{X} = \frac{-3 + (-1) + 1 + 3}{4} = 0, \quad \bar{Y} = \frac{0 + 1 + 2 + 4}{4} = 1.75
\]

次に、傾き \(a\) と切片 \(b\) を求めます。傾き \(a\) は次の式で与えられます。

\[
a = \frac{\sum (X_i - \bar{X})(Y_i - \bar{Y})}{\sum (X_i - \bar{X})^2}
\]

具体的に計算すると、

\[
\begin{aligned}
a &= \frac{(-3 - 0)(0 - 1.75) + (-1 - 0)(1 - 1.75) + (1 - 0)(2 - 1.75) + (3 - 0)(4 - 1.75)}{(-3 - 0)^2 + (-1 - 0)^2 + (1 - 0)^2 + (3 - 0)^2} \\
&= \frac{(-3)(-1.75) + (-1)(-0.75) + (1)(0.25) + (3)(2.25)}{9 + 1 + 1 + 9} \\
&= \frac{5.25 + 0.75 + 0.25 + 6.75}{20} \\
&= \frac{13}{20} \\
&= 0.65
\end{aligned}
\]

切片 \(b\) は次の式で求められます。

\[
b = \bar{Y} - a \cdot \bar{X} = 1.75 - 0.65 \times 0 = 1.75
\]

したがって、最小二乗法によって求めた直線の方程式は

\[
y = 0.65x + 1.75
\]

となります。

データ点と最小二乗法で得られた直線をグラフに示します。

\begin{figure}[H]
    \centering
    \includegraphics[width=0.7\linewidth]{linear_fit.png}
    \caption{データ点と最小二乗法による直線フィット}
\end{figure}

\newpage  % 次のページに進む

\section*{2. 右の回路について電圧 \(U_1, U_2, U_3\) と抵抗値 \(R_1, R_2, R_3\) が既知の場合、A 点の電圧値を導出せよ。}
\begin{figure}[H]
    \centering
    \includegraphics[width=0.5\linewidth]{2.png}
    \caption{抵抗と電圧源の接続回路}
\end{figure}

\subsection*{回答}
キルヒホッフの電流法則(KCL)をA点に適用します。A点に流れ込む電流の合計はゼロになります。

各抵抗を流れる電流を考えると、

\[
\begin{aligned}
I_1 &= \frac{U_1 - V_A}{R_1} \\
I_2 &= \frac{U_2 - V_A}{R_2} \\
I_3 &= \frac{U_3 - V_A}{R_3}
\end{aligned}
\]

これらの電流の合計はゼロなので、

\[
I_1 + I_2 + I_3 = 0
\]

これを展開すると、

\[
\frac{U_1 - V_A}{R_1} + \frac{U_2 - V_A}{R_2} + \frac{U_3 - V_A}{R_3} = 0
\]

両辺を整理すると、

\[
\left( \frac{1}{R_1} + \frac{1}{R_2} + \frac{1}{R_3} \right) V_A = \frac{U_1}{R_1} + \frac{U_2}{R_2} + \frac{U_3}{R_3}
\]

したがって、A点の電位 \(V_A\) は次の式で与えられます。

\[
V_A = \frac{\dfrac{U_1}{R_1} + \dfrac{U_2}{R_2} + \dfrac{U_3}{R_3}}{\dfrac{1}{R_1} + \dfrac{1}{R_2} + \dfrac{1}{R_3}}
\]

または、コンダクタンス \(G_i = \dfrac{1}{R_i}\) を用いて、

\[
V_A = \frac{G_1 U_1 + G_2 U_2 + G_3 U_3}{G_1 + G_2 + G_3}
\]

この式により、既知の電圧と抵抗値からA点の電圧を計算することができます。

\newpage  % 次のページに進む

\section*{3. 下記オペアンプ回路について電圧 \(U_{\text{IN}}\) と抵抗値 \(R_1, R_2, R_3\) が既知の場合、抵抗 \(R_3\) に流れている電流 \(I_3\) を導出せよ。\(R_1=10\text{k}\Omega, R_2=50\text{k}\Omega\) の際、回路の増幅率を計算せよ。}
\begin{figure}[H]
    \centering
    \includegraphics[width=0.6\linewidth]{3.png}
    \caption{反転増幅回路}
\end{figure}

\subsection*{回答}

まず、反転増幅回路の増幅率(電圧利得)を求めます。

\[
\text{増幅率} = A_v = -\frac{R_2}{R_1}
\]

抵抗値が \(R_1 = 10\ \text{k}\Omega\), \(R_2 = 50\ \text{k}\Omega\) の場合、

\[
A_v = -\frac{50\ \text{k}\Omega}{10\ \text{k}\Omega} = -5
\]

したがって、出力電圧 \(V_{\text{out}}\) は入力電圧 \(U_{\text{IN}}\) を用いて、

\[
V_{\text{out}} = A_v \times U_{\text{IN}} = -5 \times U_{\text{IN}}
\]

次に、抵抗 \(R_3\) に流れる電流 \(I_3\) を求めます。オペアンプの出力電圧は \(V_{\text{out}}\) であり、抵抗 \(R_3\) のもう一方の端は接地されています。

したがって、\(I_3\) は

\[
I_3 = \frac{V_{\text{out}} - 0}{R_3} = \frac{-5 U_{\text{IN}}}{R_3}
\]

つまり、

\[
I_3 = -\frac{5 U_{\text{IN}}}{R_3}
\]

となります。負の符号は、電流の向きが出力端子から抵抗 \(R_3\) を通って接地へ流れることを示しています。

\newpage  % 次のページに進む

\section*{Ex.1 データ \((x_i, y_i), i=1,2,\dots,n\) を最小2乗法によって下記式に当てはまる場合の係数(\(a, b, c\))を導出せよ。}
\begin{quote}
\( y = ax^2 + bx + c \)
\end{quote}

\begin{figure}[H]
    \centering
    \includegraphics[width=0.7\linewidth]{e1.png}
    \caption{データ点のプロット}
\end{figure}

\subsection*{回答}
データに対して2次関数 \( y = ax^2 + bx + c \) を最小二乗法でフィッティングします。

まず、正規方程式を立てます。

\[
\begin{cases}
\sum y_i = a \sum x_i^2 + b \sum x_i + c n \\
\sum x_i y_i = a \sum x_i^3 + b \sum x_i^2 + c \sum x_i \\
\sum x_i^2 y_i = a \sum x_i^4 + b \sum x_i^3 + c \sum x_i^2
\end{cases}
\]

データから各和を計算します。

データ:

\[
\begin{array}{c|c|c|c|c|c|c|c}
i & x_i & y_i & x_i^2 & x_i^3 & x_i^4 & x_i y_i & x_i^2 y_i \\
\hline
1 & x_1 & y_1 & x_1^2 & x_1^3 & x_1^4 & x_1 y_1 & x_1^2 y_1 \\
\vdots & \vdots & \vdots & \vdots & \vdots & \vdots & \vdots & \vdots \\
n & x_n & y_n & x_n^2 & x_n^3 & x_n^4 & x_n y_n & x_n^2 y_n \\
\end{array}
\]

これらの値を具体的に計算し、正規方程式を解くことで係数 \(a\), \(b\), \(c\) を求めます。

計算の結果、係数は次のようになります。

\[
a = 0.06, \quad b = 0.65, \quad c = 1.44
\]

したがって、最小二乗法による2次関数は

\[
y = 0.06x^2 + 0.65x + 1.44
\]

となります。

フィッティング結果をデータ点とともにプロットします。

\begin{figure}[H]
    \centering
    \includegraphics[width=0.7\linewidth]{quadratic_fit.png}
    \caption{データ点と最小二乗法による2次関数フィット}
\end{figure}

\newpage  % 次のページに進む

\section*{Ex.2 右の回路について電圧 \(U_A, U_B, U_C\) と抵抗値 \(R_1, R_2, R_3, R_4, R_5\) が既知の場合、AB 点間の電圧値 \(E\) を導出せよ。}
\begin{figure}[H]
    \centering
    \includegraphics[width=0.6\linewidth]{e2.png}
    \caption{複合回路}
\end{figure}

\subsection*{回答}
回路内の電流と電圧の関係を用いて、AB間の電圧 \(E\) を求めます。

まず、抵抗 \(R_1\), \(R_2\), \(R_3\) を介して点AとBに接続された電圧源 \(U_A\), \(U_B\), \(U_C\) があるとします。

キルヒホッフの電流法則より、点ABにおける電流の合計はゼロです。

\[
\frac{U_A - E}{R_1} + \frac{U_B - E}{R_2} + \frac{U_C - E}{R_3} = 0
\]

これを整理すると、

\[
\left( \frac{1}{R_1} + \frac{1}{R_2} + \frac{1}{R_3} \right) E = \frac{U_A}{R_1} + \frac{U_B}{R_2} + \frac{U_C}{R_3}
\]

したがって、

\[
E = \frac{\dfrac{U_A}{R_1} + \dfrac{U_B}{R_2} + \dfrac{U_C}{R_3}}{\dfrac{1}{R_1} + \dfrac{1}{R_2} + \dfrac{1}{R_3}}
\]

コンダクタンス \(G_i = \dfrac{1}{R_i}\) を用いると、

\[
E = \frac{G_1 U_A + G_2 U_B + G_3 U_C}{G_1 + G_2 + G_3}
\]

これにより、AB間の電圧 \(E\) を求めることができます。

\newpage  % 次のページに進む

\section*{Ex.3 下記オペアンプ回路について電圧 \(U_{\text{IN}}\) と抵抗値 \(R_1, R_2, R_3\) が既知の場合、オペアンプの出力端子に流れ込む電流 \(I_o\) を導出せよ。}
\begin{figure}[H]
    \centering
    \includegraphics[width=0.6\linewidth]{e3.png}
    \caption{オペアンプ回路}
\end{figure}

\subsection*{回答}
オペアンプの理想動作を仮定すると、入力端子間の電位差はゼロ(仮想短絡)であり、入力端子に流れ込む電流もゼロ(無限大入力インピーダンス)となります。

まず、ノード電圧法を用いて出力電圧 \(V_{\text{out}}\) を求めます。

反転入力端子の電位を \(V_-\) とすると、\(V_- = V_+\) ですが、非反転入力端子は接地されているため、\(V_+ = 0\) です。したがって、\(V_- = 0\) となります。

入力側の電流:

\[
I_1 = \frac{U_{\text{IN}} - V_-}{R_1} = \frac{U_{\text{IN}} - 0}{R_1} = \frac{U_{\text{IN}}}{R_1}
\]

フィードバック側の電流:

\[
I_2 = \frac{V_{\text{out}} - V_-}{R_2} = \frac{V_{\text{out}} - 0}{R_2} = \frac{V_{\text{out}}}{R_2}
\]

オペアンプの入力端子に流れ込む電流はゼロなので、\(I_1 + I_2 = 0\) となります。

\[
\frac{U_{\text{IN}}}{R_1} + \frac{V_{\text{out}}}{R_2} = 0
\]

これを \(V_{\text{out}}\) について解くと、

\[
V_{\text{out}} = -\frac{R_2}{R_1} U_{\text{IN}}
\]

出力電流 \(I_o\) は、出力電圧 \(V_{\text{out}}\) から抵抗 \(R_3\) を通して流れます。

\[
I_o = \frac{V_{\text{out}} - 0}{R_3} = \frac{-\dfrac{R_2}{R_1} U_{\text{IN}}}{R_3} = -\frac{R_2 U_{\text{IN}}}{R_1 R_3}
\]

したがって、オペアンプの出力端子に流れ込む電流 \(I_o\) は

\[
I_o = -\frac{R_2 U_{\text{IN}}}{R_1 R_3}
\]

となります。負の符号は、電流の向きがオペアンプの出力端子から抵抗 \(R_3\) を通って接地へ流れることを示しています。

\end{document}
